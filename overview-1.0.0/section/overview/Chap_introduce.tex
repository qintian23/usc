\section{引言}

对于可预见的未来,设计图像检索工具的主要限制是我们对视觉的理解十分有限。尽管理解不充分,我们也能构造有用的工具,就如IBM的已经出现在大量的市场广告中的图像搜素产品QBIC,以及看上去生意兴隆的
Virage 公司的图像搜索引擎等等,但是我们仍然很难去评价怎样算成功。
表示图像的方式粗略地分有三种:在像素级,人们对具体的箱数值感兴趣;在组合级,人们关心图像的整体外观;或是在对象语义级,人们关注图像所描述的事务。

\subsection{全局准则}

对于可预见的未来,设计图像检索工具的主要限制是我们对视觉的理解十分有限。尽管理解不充分,我们也能构造有用的工具,就如IBM的已经出现在大量的市场广告中的图像搜素产品QBIC,以及看上去生意兴隆的
Virage 公司的图像搜索引擎等等,但是我们仍然很难去评价怎样算成功。
表示图像的方式粗略地分有三种:在像素级,人们对具体的箱数值感兴趣;在组合级,人们关心图像的整体外观;或是在对象语义级,人们关注图像所描述的事务。

\subsection{局部特征准则}

对于可预见的未来,设计图像检索工具的主要限制是我们对视觉的理解十分有限。尽管理解不充分,我们也能构造有用的工具,就如IBM的已经出现在大量的市场广告中的图像搜素产品QBIC,以及看上去生意兴隆的
Virage 公司的图像搜索引擎等等,但是我们仍然很难去评价怎样算成功。
表示图像的方式粗略地分有三种:在像素级,人们对具体的箱数值感兴趣;在组合级,人们关心图像的整体外观;或是在对象语义级,人们关注图像所描述的事务。

\subsection{特征学习准则}

对于可预见的未来,设计图像检索工具的主要限制是我们对视觉的理解十分有限。尽管理解不充分,我们也能构造有用的工具,就如IBM的已经出现在大量的市场广告中的图像搜素产品QBIC,以及看上去生意兴隆的
Virage 公司的图像搜索引擎等等,但是我们仍然很难去评价怎样算成功。
表示图像的方式粗略地分有三种:在像素级,人们对具体的箱数值感兴趣;在组合级,人们关心图像的整体外观;或是在对象语义级,人们关注图像所描述的事务。