\section{引言}

元宇宙概念在近年来吸引了大量企业与研究者的关注,主要是因为其在技术上,特别是与虚拟图像、计算机辅助技术领域的三大突破有关。其一是英伟达发布了世界首款实时光线追踪GPU,
我们知道高质量的3D渲染的核心算法是基于几何光学的光线追踪法。二十年前,该算法只能在昂贵的Sun或者SGI工作站上运算。依随岁月的流逝,越来越多的物理定则被加入到算法流程之
中,渲染效果愈发逼真。几乎所有的电影特效都是基于光学追踪法,一部电影往往需要数千台Linux服务器计算数年。长久以来,大家都将实时光线追踪计算作为一个梦想。终于,英伟达的GPU技术积累到达了这个临界点。

其二便是Epic Game发布的虚幻引擎五,它具备两大全新核心技术:Nanite虚拟微多边形几何技术和Lumen动态全局光照技术。Nanite虚拟几何技术的出现意味着由数以亿计的多边形组
成的影视级艺术作品可以被直接导入虚幻引擎,Nanite几何体可以被实时流送和缩放,因此无需再考虑多边形数量预算、多边形内存预算或绘制次数预算了;也不用再将细节烘焙到法线
贴图或手动编辑细节层次(LOD),这必定是图形学领域革命性的飞跃。

其三便是AI的GAN model,对抗生成网络(Generative Adersarial Network GAN)获得了爆炸式的增长,其应用范围几乎涵盖了图像处理和机器视觉的绝大多数领域。其精妙独到的构思,
令人拍案叫绝;其绚烂逼真的效果,令众生颠倒。一时间对抗生成网络引发了澎湃汹涌的技术风潮,纳什均衡的概念风靡了整个人工智能领域。GAN的核心思想是构造两个深度神经网络:
判别器D和生成器G,用户为GAN提供一些真实货币作为训练样本,生成器G生成假币来欺骗判别器D,判别器D判断一张货币是否来自真实样本还是G生成的伪币;判别器和生成器交替训练,
能力在博弈中同步提高,最后达到平衡点的时候判别器无法区分样本的真伪,生成器的伪造功能炉火纯青,生成的货币几可乱真。这种阴阳互补,相克相生的设计理念为GAN的学说增添了魅力。

\subsection{全局准则}

对于可预见的未来,设计图像检索工具的主要限制是我们对视觉的理解十分有限。尽管理解不充分,我们也能构造有用的工具,就如IBM的已经出现在大量的市场广告中的图像搜素产品QBIC,以及看上去生意兴隆的
Virage 公司的图像搜索引擎等等,但是我们仍然很难去评价怎样算成功。
表示图像的方式粗略地分有三种:在像素级,人们对具体的箱数值感兴趣;在组合级,人们关心图像的整体外观;或是在对象语义级,人们关注图像所描述的事务。

\subsection{局部特征准则}

对于可预见的未来,设计图像检索工具的主要限制是我们对视觉的理解十分有限。尽管理解不充分,我们也能构造有用的工具,就如IBM的已经出现在大量的市场广告中的图像搜素产品QBIC,以及看上去生意兴隆的
Virage 公司的图像搜索引擎等等,但是我们仍然很难去评价怎样算成功。
表示图像的方式粗略地分有三种:在像素级,人们对具体的箱数值感兴趣;在组合级,人们关心图像的整体外观;或是在对象语义级,人们关注图像所描述的事务。

\subsection{特征学习准则}

对于可预见的未来,设计图像检索工具的主要限制是我们对视觉的理解十分有限。尽管理解不充分,我们也能构造有用的工具,就如IBM的已经出现在大量的市场广告中的图像搜素产品QBIC,以及看上去生意兴隆的
Virage 公司的图像搜索引擎等等,但是我们仍然很难去评价怎样算成功。
表示图像的方式粗略地分有三种:在像素级,人们对具体的箱数值感兴趣;在组合级,人们关心图像的整体外观;或是在对象语义级,人们关注图像所描述的事务。