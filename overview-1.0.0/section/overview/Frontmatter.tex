%---------------------------------------------------------------------------%
%-> Frontmatter
%---------------------------------------------------------------------------%
%-
%-> 生成封面
%-
\maketitle% 生成中文封面
%-
%-> 目录
%-
% \tableofcontents% 生成目录
%---------------------------------------------------------------------------%
%-
%-> 中文摘要
%-
\intobmk\section*{基于最优传输理论图像匹配算法文献综述}% 显示在书签但不显示在目录 

\vspace*{10pt}
   
\thispagestyle{plain}
\setcounter{page}{1}% 开始页码
\pagenumbering{Roman}% 页码符号

\noindent
\textbf{摘要}:图像匹配是虚拟图像重建的一个关键步骤,是视觉识别的一个核心过程,是图像检索的中心操作模块。视觉算法在过去几十年来飞速发展,基于图像匹配的
应用更是层出不穷,从卫星遥感图像配准到纳米级零件配准,在科学研究、社会需求、工业制造等各个方面,图像匹配这一基本问题无处不在。而对于图像匹配其定义一般被描述如下:
同一摄影项目的图像如医学图像,可以在遮挡、多姿态等条件下从任何的光照强度和频率以各种角度拍摄,这些同一项目的图像之间存在内容、结构、特征、色彩及纹理等
对应关系,而图像匹配则致力于这些关系中的一致性和相似性分析。

尤其随着计算机运算能力的提升、图像处理加速芯片的发展,图像匹配算法数量越来越多,其类型也越来越丰富。随之产生了两个问题:新方法与旧方法的取舍;如何在现有理论指导下
设计更适用、准确、鲁棒的高性能匹配算法。为了回答这两个问题,我们有必要系统的回顾和评估过去和现在的图像匹配算法。我在检索相关的文献时,发现了一个奇怪的现象:如今大热的并且在理论上
对于图像匹配算法的发展具有巨大的推进潜力的最优传输理论方面的内容常常被一些重要的文献综述\cite{ma2021image}所忽略。因此,本文主要讨论基于最优传输理论的图像匹配算法。

本文首先回顾了遵循着从手工设计特征到深度学习的图像匹配算法的发展路线,并简要分析了其中各个阶段的算法的特点。然后,详细地介绍最优传输理论在视觉算法上的
进展,其中涉及计算最优传输映射的方法、计算Wasserstein距离的方法的概述;接着,列举几个基于最优传输理论的图像匹配应用,并且与经典的解决方案相比较,以此了解OT在视觉算法中的意义。
最后,我们总结了图像匹配技术的现状,并对未来的工作进行了富有洞察力的讨论和展望。本调查可作为(但不限于)图像匹配及相关领域的研究人员和工程师参考。


\keywords{图像匹配,共形映射,最优传输理论,最佳质量传输映射,Wasserstein距离,曲面配准,手工设计特征,深度学习}% 中文关键词

%-
%-> 英文摘要
%-
\intobmk\section*{\textbf{Graduation Thesis Literature Review}}
\vspace*{20pt}

\intobmk\section*{\textbf{Graduation Thesis Topic:}  \zihao{-4}\uline{Literature review of image matching algorithms based on optimal transmission theory}}

\begin{tabular}{ll}\centering
    Student name: Zhu Liucheng     &    Student number: 20184390213 \\
    Tutor name:   Gaoyou           &    Professional qualifications: Lecturer
\end{tabular}

\vspace*{20pt}

\noindent
\textbf{Abstract} : Image matching is a key step in virtual image reconstruction, a core process of visual recognition, and a 
central operation module of image retrieval. With the rapid development of visual algorithms in the past decades, image 
matching-based applications have emerged one after another. From satellite remote sensing image registration to nano-scale parts 
registration, the basic problem of image matching is ubiquitous in scientific research, social needs, industrial manufacturing 
and so on. The definition of image matching is generally described as follows: images of the same photographic project, such as 
medical images, can be taken from any angle of light intensity and frequency under occlusion, multi-posture, etc. There are 
corresponding relationships among the images of the same project, such as content, structure, feature, color and texture,  
while image matching focuses on consistency and similarity analysis in these relationships.

Especially with the improvement of computer computing ability and the development of image processing acceleration chip, 
there are more and more image matching algorithms and their types are more and more abundant. Two problems arise: the choice 
between the new method and the old one; How to design a more applicable, accurate and robust high performance matching algorithm 
under the guidance of the existing theory. In order to answer these two questions, it is necessary to systematically review and 
evaluate past and present image matching algorithms. When I retrieve the relevant literature, I found a strange phenomenon: 
the hot and theoretically promising aspects of optimal transmission theory for the development of image matching algorithms are 
often ignored by some important literature reviews\cite{ma2021image}. Therefore, this paper mainly discusses image matching 
algorithms based on optimal transmission theory.

This paper first reviews the development of image matching algorithms that follow the path from manual design features to in-depth 
learning, and briefly analyses the characteristics of the algorithms at each stage. Then, the progress of optimal transmission 
theory in visual algorithms is introduced in detail, including an overview of the methods for calculating optimal transmission 
mappings and Wasserstein distances. Next, several image matching applications based on optimal transmission theory are listed and 
compared with classical solutions to understand the significance of OT in visual algorithms. Finally, we summarize the current 
status of image matching technology, and make insightful discussions and prospects for future work. This survey can be used as a 
reference for (but not limited to) researchers and engineers in image matching and related fields.


\KEYWORDS{Image matching, Conformal mapping, Optimal transmission theory,  Optimal mass transport mapping, Wasserstein distance, Surface registration, Handcrafted feature, Deep learing}% 英文关键词
%---------------------------------------------------------------------------%
