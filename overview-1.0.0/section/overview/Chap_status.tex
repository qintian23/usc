
\section{OT匹配算法的应用}

图像匹配是计算机视觉中的一个基本问题,被认为是广泛应用中的先决条件。其中具有代表性的主要有
1. SFM(Structure-from-motion),从一系列图像中恢复静止场景的3D结构。
2. SLAM(Simultaneous Localization and Mapping),同时定位和映射。
3. Visual Homing,旨在仅基于信息将机器人从任意起始位置导航到目标或原始位置。
4. 图像配准与变形,在\cite{haker2004optimal}中详细介绍了OT在实际成像问题中,所做的图像配准和变形的步骤,其中的一个关键之处,便是在给出了一个精确Monge-Kantorovich问题的公式
然后通过寻找保质量映射的极坐标分解的等价问题来找到最优映射[Gangbo 1994 ; makeken, 2001],可通过自然梯度下降算法来实现,然后在函数中加入一个比较项来惩罚强度的变化。这样一来,便
能适用于图像变形。又如\cite{ma2019supine}中通过改进仰卧位到俯卧位结肠的配准框架,实现了基于最优质量运输的可视化;通过在新的目标测量密度中引入高斯曲率,设计了一个基于OMT-Map的矩形域目标
测量密度,使得息肉可以被放大,从而更好地显示息肉,并使用颜色去编码距离的变化。以及在\cite{su2015optimal}中,建议使用最佳质量传输映射进行形状的匹配和比较。因为基于Monge-Brenier的OT方法
它们的归一化保形映射和OT映射是唯一的,没有重新参数化的歧义优于弹性形状度量方法。还可以定制唯一的OT映射穷尽整个图像的微分同胚组。此外有更好的适用性。

5. 图像合成,在\cite{RN1}中,通过两个不同域之间进行线性插值和翻译将联合Wasserstein自编码器应用在跨视频合成任务中。在\cite{de2011optimal}介绍了如何使用OT映射匹配和合成简化图像。
该算法将含有噪声和离群点的缺陷点集作为输入,其中输入点集被认为是Dirac测度的和,而单纯复形被认为是0-和1-单形上的一致测度的和。通过对输入点集的Delaunay三角剖分进行贪婪抽取,设计了一种由细到粗的
方案来构造所得到的单纯复形。

6. 图像检索、对象识别和跟踪,OT的匹配算法也在某些成像应用的背景下进行了研究,特别是基于内容的图像检索[Rubner1999; Rubner 1998 Levina Bickel2001]。在这项工作者,图像的像素根据其颜色位置和空间位置
被分成几个bins(或称为“签名”)。通过计算两幅图像之间的Wasserstein距离进行图像检索。在\cite{ma2015surface}中提出了一种基于Monge-Brenier的OMT理论的Wasserstein距离方法用于癫痫患者大脑海马的形状分类。
该方法首先利用保角映射将具有圆盘拓扑结构的度量曲面映射到单位平面圆盘上,然后通过诱发的面积畸变获得概率测度。通过计算具有两个概率测度的两个表面之间唯一的OT映射,可以得到定义两个
表面之间的Wasserstein距离的OT成本。该Wasserstein距离本质上可以衡量基于形状的表面之间的差异,因此可以用于形状分类。