
\section{可选的加速方案和评估方法}

\subsection{可选的加速方案}

有很多常用的加速方法可应用到计算机视觉流程中,包括监控内存管理、使用线程进行粗粒度并行、使用SIMD和SIMT方法的数据级并行、多核并行、高级的CPU和GPU汇编语言指令,硬件加速器等。有两种
基本的加速方法:
\begin{enumerate}
    \item 针对数据的加速方法;
    \item 针对算法的加速方法。
\end{enumerate}

计算设备的优化算法也称为流处理,是设计时通常会考虑的。但优化数据流和数据驻留可得到更好的结果,例如,在计算资源来回传递数据和格式化数据不是好的做法,这样的数据复制和格式转换会花很多时间
和功耗。在慢的系统内存中复制数据比通过计算单元的快速寄存器来访问数据要慢得多。这需要基于内存速度来考虑内存架构的层次,考虑计算机视觉中图像的密集型特征,最好是找到跟踪数据的方法,并将
数据尽可能长时间驻留在快速寄存器和高速缓冲区中。(可例sinkhorn)

\subsection{可选的评估方法}

对于医学图像等匹配算法评估方法,

1. 距离误差
2. 曲率差
3. 区域失真评价
4. 不同目标措施的比较
5. 目标措施效果的可视化
6. 视觉配准评估
7. 消融实验

而对于视频流上的图像匹配算法评估方法
( 1) Inception Score(IS)[38]广泛用于生成模型中.IS通过利用Inception-V3模型[3的类别预测信息来评估生成样本的质量和多样性.
(2) Frechet Inception Distance(FID)[40]也是一个广泛使用在生成模型上的指标.FID可以评估生成图像的质量,因为它能捕获生成样本与真实样本的相似性,并与人类判断相关联.
(3)Video variant of FID(FID4Video)[41]评估视频的质量和连贯性.本文使用一个预训练的视频识别模型I3D1]对视频序列提取特征.然后,对这些特征计算 FID4 Video.
一般而言,对于IS指标,值越大代表着翻译的图像质量越好﹔对于FID 和 FID4 Video这两种指标,值越小意味着翻译的图像或视频的质量越好.


\nocite{*}% 使文献列表显示所有参考文献(包括未引用文献)