
\section{总结与展望}

图像匹配在各种视觉应用中发挥了重要作用,并引起了相当大的关注。在过去的几十年里,研究人员在这一领域也取得了重大进展。因此,我们对现有的图像匹配方法(从手工制作的到可训练的)进行了全面的回顾,以便为该社区的研究人员提供更好的参考和理解。

图像匹配可以简单地分为基于区域和基于特征的匹配。基于区域的方法用于在不从图像中检测任何显着特征点的情况下实现密集匹配。它们在高重叠图像匹配(如医学图像配准)和窄基线立体(如双目立体匹配)中更受欢迎。基于深度学习的技术已经引起了对这种管道的越来越多的关注。因此,我们在 Sect 中对这些类型的方法进行了简要回顾。 4、更多地关注基于学习的方法。

基于特征的图像匹配可以有效解决大视点、宽基线和严重的非刚性图像匹配问题的局限性。它可以用于显着特征检测、判别描述和可靠匹配的管道中,通常包括转换模型估计。按照这个过程,特征检测可以从图像中提取出独特的结构。同时,特征描述可以看作是一种图像表示方法,广泛用于图像编码和相似度测量。匹配步骤可以扩展为不同类型的匹配形式,例如图匹配、点集注册、描述符匹配和不匹配去除,以及3-D情况下的匹配任务。这些比基于区域的方法更灵活和适用,从而在图像匹配领域受到相当大的关注。因此,我们以它们从传统技术到经典学习和深度学习的核心思想来回顾它们。此外,为了全面了解图像匹配的重要性,我们介绍了与图像匹配相关的几个应用。我们还通过对代表性数据集的广泛实验,对这些经典和基于深度学习的技术进行全面客观的比较和分析。

尽管在理论和性能方面都取得了长足的进步,但图像匹配仍然是一个悬而未决的问题,需要进一步努力。

文献中广泛采用的特征匹配的两阶段策略仅对具有足够相似描述符的一小组潜在对应关系执行不匹配消除。但是,这可能会导致召回性能受限,这在某些情况下可能会出现问题。

在不同的场景中,不是在不同图像中物理上相同点的投影之间寻求对应,而是在一个类别内不同实例的语义类比之间寻求对应。这需要新的范式来进行特征描述和不匹配消除中的特征匹配。

多幅图像的联合匹配已被证明可以极大地提高成对匹配的匹配性能,并且近年来引起了相当大的关注。然而,复杂性仍然是问题的主要关注点。因此,需要实用且高效的算法。

近年来,深度学习方案迅速发展,并在与计算机视觉相关的许多研究领域显示出巨大的进步。然而,在特征匹配的文献中,大多数作品都将深度学习技术应用于特征检测和描述。因此,未来可以进一步探索准确特征匹配的潜在能力。

多模态图像之间的图像匹配仍然是一个未解决的问题。未来,深度学习技术可用于更好的特征检测和描述性能。

特征匹配是计算机视觉中的一项基本任务。然而,它的应用还没有得到充分的探索。因此,一个有前途的研究方向是定制现代特征匹配技术以满足实际视觉任务的不同要求,例如 SfM 和 SLAM。

\nocite{*}% 使文献列表显示所有参考文献(包括未引用文献)