% !TeX root = ../computer version.tex

\section{摄像机}

\subsection{习题}

\begin{problem}
    试推导出于针孔前面$f'$处的虚拟图像的透视方程式投影。
\end{problem}

\begin{problem}
    试从几何上证明,谋改革平面 $\Pi$ 中两条平行线的投影汇聚到一条水平线 $H$ 上,该水平线是图像平面$\Pi$ 与过针孔点平行于的平面交线。
\end{problem}

\begin{problem}
    用透视投影式(1.1)从代数上证明与上题相同的内容。为了简单起见,可以假设改平面$\Pi$与图像平面平行。
\end{problem}

\begin{problem}
    试用Snell规则说明过薄透镜中光心的射线没有折射现象,并推导薄透镜方程。提示:考虑一条过点$P$的射线$r_0$,并分别构造透镜的右轮廓和左轮廓对$r_0$折射而得到的两条射线$r_1$与$r_2$。
\end{problem}

\begin{problem}
    考虑一个用薄透镜配备的摄像机,图像平面在$z'$位置,而平面上的景物点聚焦在$z$处。现假设图像平面移动至$\hat{z} '$,证明相应的模糊圆的直径为
    $$\mathrm{d}\frac{\left | z' - \hat{z}'  \right | }{z'}  $$
    其中,$d$是透镜的直径。使用以上结果来说明视场深度(也就是使模糊圆的直径低于某个阈值$\varepsilon$的最近与最远平面之间的距离)可按下式计算
    $$ D=2\varepsilon fz(z+f)\frac{d}{f^2d^2-\varepsilon ^2z^2} $$
    并且做出结论,即对一个固定的焦距长度,视场深度随透镜直径减少而增加,$f$数也因而增加。
    提示:解出图像聚焦在图像平面上$\hat{z}'$位置的点的深度$\hat{z}$;要考虑$z'$比$\hat{z}'$大与小两种情况。
\end{problem}

\begin{problem}
    在一个薄透镜的两个焦点分别为$F$与$F'$的条件下,拥挤和方法构造点$P$的图像$P'$。
\end{problem}

\begin{problem}
    推出厚透镜两个球面边半径相同的厚透镜方程。
\end{problem}