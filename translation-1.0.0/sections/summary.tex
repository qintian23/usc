% !Tex
\quad
\title{
	\heiti{
		对深度学习的几何理解
	}
}
\date{}
\maketitle
\thispagestyle{fancy}            %更改plain状态,首页格式设为fancy
\renewcommand{\headrulewidth}{1pt}      %把页眉线的宽度设为零,即去掉页眉线
%\vspace{3mm}
%\renewcommand{\abstractname} {} % 不显示摘要名字

%\begin{abstract}
	\xiaosihao{}
%	\song{}
	\noindent
%	\raggedright
	\textbf{摘要:}
	这篇论文详细地给出了在深度学习中如何运用最优传输理论的相关知识的方法与技巧。这项工作引入了生成式对抗网络(GANs)的最优运输(OT)视图。自然数据集具有内在的模式,可以概括为流形分布原理:一类数据的分布接近于一个低维流形。GANs主要完成两个任务:流形学习和概率分布转换。后者可以用经典的OT方法来进行。从OT的角度来看,生成器计算OT图,而鉴别器计算生成的数据分布和真实数据分布之间的瓦瑟斯坦距离;两者都可以简化为凸几何优化过程。此外,OT理论还发现了发生器和鉴别器之间存在内在的协作关系,而不是竞争关系,以及模态崩溃的根本原因。我们还提出了一种新的生成模型,该模型使用自编码器(AE)进行流形学习,并使用OT映射进行概率分布变换。该AE-OT模型提高了理论的严谨性和透明度,以及计算的稳定性和效率;特别是,它消除了模式崩溃。实验结果验证了我们的假设,并证明了我们所提出的模型的优点。
	\\
	\\
	\textbf{关键词:} 生成性;对抗性;深度学习;最优传输;模式崩溃
%\end{abstract}