% !TeX root = ../translation.tex

\section{结论}

这项工作使用OT理论来解释gan。根据数据流形分布假设,GANs主要完成两个任务:流形学习和概率分布变换。后一种任务可以直接使用OT方法来完成。这一理论理解解释了模态崩溃的根本原因,并表明了发生器和鉴别器之间的内在关系应该是协作,而不是竞争。此外,我们提出了一个AE-OT模型,它提高了理论的严密性、训练的稳定性和效率,并消除了模式崩溃。

我们的实验验证了我们的假设,即如果分布运输图是不连续的,那么奇点集的存在会导致模态崩溃。此外,当我们提出的模型与现有的技术进行比较时,我们的方法消除了模态崩溃,并在FID评分和PRD曲线方面优于其他模型。

在未来,我们将探索对流形学习阶段的理论理解,并使用一种严格的方法来使黑盒的这一部分透明。