% !TeX root = ../uvanode.tex

\section{国内外研究现状和发展动态}

\subsection{国内外研究现状}

多智能体在生态信息系统、人工神经网络、经济网络、小世界网络等研究中都有出现,在国民经济和国防科技领域中,诸多研究都可归结于集群、多智能体协同工作的复杂系统(如无人驾驶地面交通系统,卫星群系统,无人机编队等)。在众多的模型中都应用归功于匈牙利科学院院士Vicsek在1995年提出了著名的Vicsek模型[1]。该模型从微观角度来刻画平面物理粒子之间相互影响和具有自我组织及自我演化特征。Vicsek模型是一个经典的离散数学模型,它也适合用来描述鸟群、蜂群、移动电话网络等群体行为。随后,Jadbabaie[2]给出了该模型的解析行为描述。在引入邻接系数函数后,Cucker 和Smale [3]给出了Vicsek 模型的集群特征描述,并给出了对应的连续性模型。Vicsek 模型假设所有个体的耦合强度相同,而Cucker -Smale 模型描述的是在系统内假设每个粒子能够调整自身的速度来匹配其他邻近的粒子。因而它们的主要区别在于Vicsek 模型中任意两个粒子要么存在联系要么不存在,具有局部信息传递特征,而在Cucker -Smale模型里任意两个粒子都存在联系,只不过是这两个粒子之间的信息传递是以通信速率函数来刻画,并随着粒子间距离的增大而衰减的。这期间Motsch 和Tadmor[4]考虑了非对称条件下的集群分析并给出了集群特征的数学刻画。 近几年里,对多智能体的一致性问题的研究又有了突飞猛进的发展,已经形成了若干个研究热点,如具有随机干扰[5],延迟的一致性问题[6]、免碰撞的一致性问题[7]、具有虚拟领导者的一致性问题[8]、刻画群体大小[9], 双边集群问题[10]等等。在应用方面,学者们也取得了一系列的研究结果,如针对 UAV 对目标的跟踪问题,文献[11]给出了导航向量场的构建步骤,并阐述了UAV 编队协同跟踪目标的控制条件,解决 UAV 编队在协同追踪目标过程中避障问题的有效性。文献[12]通过矩阵分析理论和线性系统李雅普诺夫稳定性分析理论进行分析,获得了无人机集群系统达成期望编队队形系统的充分条件。作者[13]基于ESO 的输出提出了抗扰编队控制律,并提出一套算法来对控制律进行参数选定,建立多无人机系统实现抗扰时变编队所需要的充要条件,并最终严格证明了在满足编队充要条件和基于提出的控制律下,多无人机系统可以稳定实现抗扰时变编队。文献[14]以一致性理论为基础,针对无人机运动模型的特点与实际飞行要求,对基本的一致性算法进行改进,提出了改进一致性无人机编队控制算法,提出了“最小调整”约束条件处理策略,依据粒子群算法对各无人机的爬升加速度进行优化,解决了避免碰撞问题。

\subsection{发展动态}

当今世界,一方面,随着复杂系统的控制理论和应用技术的日趋成熟,多智能体在生态信息系统、人工神经网络、经济网络、小世界网络等研究中频频出现,应用前景十分广泛。并且多智能体系统的一致性问题被广泛的应用于编队控制问题、聚集问题、同步以及协调决策问题等研究中。与此同时,随着科技进步和社交网络发展,伴随而来的是越来越多的复杂现象和任务。面对越来越复杂的任务,单个智能体已无法完成任务,由此使多个智能体构建成一个群体,协同完成某一特定复杂的任务成为必然。例如,在国民经济和国防科技领域中,诸多研究都可归结于集群、多智能体协同工作的复杂系统(如无人驾驶地面交通系统,卫星群系统,无人机编队等)。另一方面,多智能体系统研究是一新兴的研究领域,还有诸多的问题亟待研究解决,比如说关于多智能体系统在有限时间下的集群性和编队方面的问题研究等。