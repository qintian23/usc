% !TeX root = ../uvanode.tex

\section{研究目的}

本项目旨在对具外部指令条件下多智能体系统在有限时间内的集群性及其在多无人机协同编队中的应用。所涉及的集群性分析属于微分方程与动力系统范畴,是经典定性、稳定性理论在复杂系统中的具体应用,并会将所获的最近成果运用于军事领域——多无人机协同编队建模、仿真和分析等方面的应用研究。同时通过本项目的研究,一方面可为研究多智能体系统提供新方法。另一方面,使人为干预多智能体系统的集群模式定量化, 实现对集群性质的人为控制。从而为有限时间内的多无人机协同编队、卫星编队、水雷布阵等军事应用提供坚实的数学理论基础。本项目的研究目的分为以下两个方面:

在理论上,旨在开辟具有外部指令影响的多智能系统的建模和分析方法,刻画具有指令体参与的多智能体系统在有限时间限制下的集群性问题,揭示系统的集群速度、集群模式等的影响规律,给出多智能体群快速集结成给定模式的收敛时间上限的参数预估值,实现对集群性质的人为控制。

在应用研究上,建立基于接受指令和有限时间限制的条件下多无人机直线编队自组织模型,通过理论分析和数值仿真,揭示多无人机协同运动中直线编队的最终状态、演变规律保证无人机之间避免碰撞等。这些研究将拓展自组织系统理论及应用的新领域,为自组织系统在军事科技领域中的应用研究提供坚实的数学理论基础。
