% !TeX root = ../uvanode.tex

\section{研究内容}

本项目以具有外部指令的多无人机协同直线编队的研究为主线,围绕收敛速度与群体规模之间存在关系等问题开展研究和数值仿真,揭示外部指令、初始状态对上述问题的影响规律。通过建立合适的数学模型及对数学模型的理论分析,研究多智能体系统的集群性,将为“无序、分散的多个智能体快速集结成有序、协同的群体并完成特定任务”提供有力的理论支撑。本项目所指的集群性是指群体在运动演化过程中的个体间的相对距离始终保持一致有界,即保持群体状态或协同性,每个个体的速度最终趋于同一速度,即集群速度。基于时限性大量地存在于现实问题之中,所以考虑有限时间内的集群问题更切合实际。如多无人机协同作战时必须考虑有限性。在少量指令和有限时间条件下,多无人机如何自主协同运动,如何快速自主集结或布阵并完成战斗编队或特定搜救和探测的任务,都是亟待解决的科学问题。本项目的主要研究内容为多无人机协同直线编队的形成机制。