\title{配准和变形的最佳质量传输}
\author{STEVEN HAKER \\ LEI ZHU AND ALLEN TANNENBAUM \\ SIGURD ANGENENT}
\date{\empty} 
\maketitle

\thispagestyle{empty}

\noindent
\textbf{摘要}:图像配准是在可能在不同时间拍摄的两个或多个图像数据集之间建立一个共同的几何参考系的过程。
本文提出了一种基于蒙格-坎托洛维奇理论计算弹性配准和翘曲映射的方法。这种质量传输方法有许多重要的特点。
首先,它是无参数的。此外,它利用了两幅图像中所有的灰度数据,将两幅图像放在相同的基础上,并且是对称的:从图像A到图像B的最优映射是从图像B到图像A的最优映射的倒数。
该方法不要求指定地标,并且所涉及的距离函数的最小化器是唯一的;没有其他的局部最小化器。最后,最佳输运自然考虑了由面积或体积变化导致的密度变化。虽然最优输运方法肯定不适用于所有的配准和翘曲问题,但这种质量保持特性使得蒙格-坎托洛维奇方法对一类有趣的翘曲问题非常有用,正如我们在本文中所展示的那样。我们寻找配准映射的方法是基于偏微分方程方法,在质量保持约束下最小化坎托洛维奇-瓦瑟斯坦或“地球移动者”的距离。
我们展示了这种方法如何导致实际的算法,并用一些例子演示了我们的方法,包括那些来自医学领域的例子。我们还扩展了这种方法,以考虑到强度的变化,并表明它非常适合于图像变形等应用。
\\
\\
\noindent
\textbf{关键词:}弹性配准,图像翘曲,最佳传输,质量保存,梯度流动