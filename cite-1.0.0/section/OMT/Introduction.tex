\section{引言}\label{section : 1}

\subsection{图像配准}

图像配准和变形是关键的挑战,必须解决的一些实际的成像问题。配准是在两个或多个图像数据集之间建立一个共同的几何参考系的过程。在医学成像的背景下,配准允许合并术前图像信息,以改善图像引导的手术和治疗。它还通过允许同时使用来自多个数据集的信息来帮助诊断,可能在不同的时间使用不同的模式和患者的位置。

注册过程分几个步骤进行。通常,建立了数据集之间的相似性度量,这样人们就可以在应用转换后量化一个图像与另一个图像的接近程度。这种度量可能包括像素强度值之间的相似性,以及预定义的图像特征的接近性,如植入的基准、解剖标志、表面轮廓和山脊线。接下来,找到使变换后的图像之间的相似性最大化的变换。通常,这种转换是作为一个优化问题的解,其中要考虑的转换被限制为一个预先确定的类。最后,一旦得到一个最优变换,它就被用来融合图像数据集。

配准有一个广泛的文献致力于它的许多方法,从统计到计算流体动力学到各种类型的扭曲方法。参见Toga(1999)最近关于这个主题的论文以及广泛的参考文献。我们在这里只回顾了一些更相关的问题。我们的方法是基于连续介质体和流体力学的翘曲策略,其中人们试图利用弹性材料的性质来确定变形。
一个定义了(典型的二次)成本函数,惩罚变形模板和目标之间的不匹配(克里斯滕森等,1993,1996;米勒等;布罗尼尔森和格兰科,1996;蒂里昂,1995)。从这个意义上说,我们的方法是最接近这些作品的注册哲学的。事实上,L2蒙格-坎托洛维奇的最优翘曲图可以看作是速度向量场,它使欧拉连续性(质量保持)方程下的标准能量积分最小化(Benamou和Brenier,2000)。
具体说明见下文第3.6节。特别是在流体力学框架中,这意味着最优Monge–Kantorovich解是作为势流给出的。

\subsection{最优传输}

本文描述的方法是为了弹性配准,并基于以l2坎托洛维奇-瓦瑟斯坦距离作为相似度量建立的优化问题。我们对所考虑的变换所施加的约束是它们服从质量保存的性质。因此,我们将在这种方法中匹配质量密度,这可以被认为是二维的加权区域或三维的加权体积。这种最优的大众运输问题最早是由加斯帕·蒙格在1781年提出的,它涉及到找到在最小的运输成本的意义上,将一堆土壤从一个地点转移到另一个地点的最佳方法。因此,坎托洛维奇-瓦瑟斯坦的距离也通常被称为“地球搬运工的距离”。这个问题在坎托洛维奇(1948)的著作中被给出了一个现代的表述,因此现在被称为蒙格-坎托洛维奇问题。

我们对蒙格-坎托洛维奇问题的兴趣最初源于我们在医疗应用方面的工作。这个问题发生在医学成像中,例如,在功能性Mr中,人们可能想要比较随时间变形的各种特征的活动程度,并获得相应的弹性配准图。这个问题的一个特殊情况发生在任何考虑体积或区域保留映射的应用程序中。例如,正如我们将在第4节中所展示的,我们的方法提供了一种方法来获得规则的保持区域的表面微分同态。我们发现这种技术对脑表面和结肠表面扁平化等应用很有用(Angenent等,1999b;Haker等,2000)。然而,我们的最优运输方法可能不适合在某些情况下当质量保存假设可能是无效的,如匹配两个不同的透视投影的空间对象,或核磁共振图像的注册宠物图像强度视为质量密度。

最优运输方法出现在计量经济学、流体动力学、自动控制、运输、统计物理、形状优化、专家系统和气象学(Rachev和R¨乌森多夫,1998)。它们自然也适用于计算机视觉中的某些问题。特别是对于一般的跟踪问题,一个健壮可靠的目标和形状识别系统至关重要。实现这一点的关键方法是通过模板匹配,即在给定的对象目录中,将某个对象与另一个对象进行匹配。通常情况下,匹配并不精确,因此需要一些形状度量来衡量对象之间的“拟合优度”或相似性(Haralick和Shapiro,1992;Fry,1993)。

在某些成像应用的背景下,最优传输问题也进行了研究,特别是基于内容的图像检索(Rubner,1999;Rubner等人,1998;Levina和Bickel,2001)。在这项工作中,图像中的像素根据它们在颜色和/或空间位置上的位置被分成几个箱子(称为“签名”)。在两幅图像的签名之间计算地球移动器的距离(EMD),然后用于图像检索。然而,这种EMD方法并没有给出在每个像素位置定义的扭曲网格或位移,这对于图像配准和图像变形是必不可少的。

利用坎托罗维奇-瓦瑟斯坦距离进行图像配准和翘曲有许多优点。它是无参数的。它利用了两幅图像中的所有灰度数据,并将两幅图像置于相同的位置。因此,它是对称的,从图像A到图像B的最优映射是从B到图像A的最优映射的倒数。不需要像克里斯滕森和约翰逊(2001)那样使用额外的约束来保证对称性质。它不要求指定地标。所涉及的距离函数的最小值器是唯一的;没有其他的局部最小化器。最后,它是专门设计来考虑由面积或体积的变化所导致的密度的变化。这最后一点对于某些应用程序是必不可少的。作为一个例子,我们在第4节中展示了如何使用该方法从保角映射中推导出正则的保面积曲面微分同构。

本文的关键贡献之一是引入了一种有效的偏微分方程方法来计算该度量和翘曲映射。我们对这个问题的解决方案允许人们从一个相当简单的一阶偏微分方程计算最优翘曲,而不是在这些工作中提出的高阶方法(克里斯坦ensen等,1993,1996)和基于线性规划的计算复杂离散方法。这有助于实现下文第4节所述的简易性和速度。我们给出了下面Monge–Kantorovich问题的精确公式(见第2节),然后开发了我们的算法。其核心思想是通过寻找保质量映射的极因子分解的等价问题来寻找最优映射(Gangbo,1994;McCann,2001)。事实证明,这可以通过自然梯度下降技术来实现。详情见第3节。在第4节中,我们在一些合成密度和真实图像上展示了我们的结果。

可以在函数中添加一个比较项来惩罚强度的变化。新功能适用于图像变形等应用,其中褪色和褪色效果是不需要的。第3节给出了相应的梯度下降技术,第4节给出了基于真实图像的两个例子。最后,我们注意到我们的方法可能是严格合理的;参见Angenent等人(2003)的数学细节。
