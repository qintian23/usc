\section{引言}\label{section : 1}

\subsection{图像配准}

图像配准和变形是关键的挑战,必须解决的一些实际的成像问题。配准是在两个或多个图像数据集之间建立一个共同的几何参考系的过程。在医学成像的背景下,配准允许合并术前图像信息,以改善图像引导的手术和治疗。它还通过允许同时使用来自多个数据集的信息来帮助诊断,可能在不同的时间使用不同的模式和患者的位置。

注册过程分几个步骤进行。通常,建立了数据集之间的相似性度量,这样人们就可以在应用转换后量化一个图像与另一个图像的接近程度。这种度量可能包括像素强度值之间的相似性,以及预定义的图像特征的接近性,如植入的基准、解剖标志、表面轮廓和山脊线。接下来,找到使变换后的图像之间的相似性最大化的变换。通常,这种转换是作为一个优化问题的解,其中要考虑的转换被限制为一个预先确定的类。最后,一旦得到一个最优变换,它就被用来融合图像数据集。

配准有一个广泛的文献致力于它的许多方法,从统计到计算流体动力学到各种类型的扭曲方法。参见Toga(1999)最近关于这个主题的论文以及广泛的参考文献。我们在这里只回顾了一些更相关的问题。我们的方法是基于连续介质体和流体力学的翘曲策略,其中人们试图利用弹性材料的性质来确定变形。
一个定义了(典型的二次)成本函数,惩罚变形模板和目标之间的不匹配(克里斯滕森等,1993,1996;米勒等;布罗尼尔森和格兰科,1996;蒂里昂,1995)。从这个意义上说,我们的方法是最接近这些作品的注册哲学的。事实上,L2蒙格-坎托洛维奇的最优翘曲图可以看作是速度向量场,它使欧拉连续性(质量保持)方程下的标准能量积分最小化(Benamou和Brenier,2000)。
具体说明见下文第3.6节。特别是在流体力学框架中,这意味着最优Monge–Kantorovich解是作为势流给出的。
