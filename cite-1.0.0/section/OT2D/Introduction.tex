\section{引言}

从散乱的点集重建形状是几何处理中的一个基本问题:
尽管取得了重大进展,但其固有的不适定性和几何数
据集增加的异质性使得当前的方法仍然远远不能令人
满意。即使是该问题的2D实例,即平面中的形状重建,
在包括计算机视觉和图像处理的各种应用领域中仍然
是一个挑战。二维点集通常从传感器获取或从图像中
提取,因此经常受到混叠、噪声和离群点的阻碍。此
外,基于图像的点集通常表现出各种各样的特征,例
如拐角、交点、分叉和边界。噪声、异常值以及边界
和特征的存在的这种组合使得大多数众所周知的策略
(包括泊松、Delaunay或基于MLS的方法)存在缺陷。

形状重建也与形状简化密切相关。虽然一些作者(特
别是在计算几何中)将重建问题限制为寻找所有输入
点的连通性,但噪声的存在和大多数数据集的绝对大
小要求最终重建的形状比输入更简洁。然而,重建和
简化通常是顺序执行的,而不是协调执行的。

相反,我们通过一个统一的框架来共同解决二维形状
的重建和简化问题,该框架植根于测量的最优传输。
具体效益包括:(I)稳健-对大量噪声和异常值的敏感性;(II)保留明显特征;
(三)边界保护;以及(IV)保证输出是(可能是非
流形的)嵌入单纯复形。

\subsection{以前的工作}

为了激发我们的方法并强调它如何满足理论和实践的
需要,我们首先回顾了以前在二维点集的重建和简化
方面的工作。

重建。对于无噪声数据集,现有的重建方法大多基于
采样假设而变化。可以使用图像细化[MIB01]、阿尔法
形状[EKS83]或R-规则形状[ATT97]来处理均匀采样的
数据集;对于非均匀采样,大多数可证明正确的方法
依赖于Delaunay滤波[ABE98],并改进了计算效率
[GS01]、采样边界[DK99,DMR99]以及拐角和开放曲线
的处理[FR01,DW02]。

在过去的十年中,噪声数据集已经通过各种方法进行
了处理[LEE00,CFG∗03,MD07,MTSM10]。

最近的成功包括自交曲线的提取[RVC11]。这些重建方
法中的大多数首先通过聚类、细化或平均来执行噪声
去除,这通常导致特征的显著钝化。从数据聚类
[SON10] 和 稳 健 统 计 [FCOS05] , 到 k 阶 阿 尔 法 形 状
[KP08]、谱方法[KSO04]和p1最小化[ASGCO10],对异常
值的稳健性也进行了较小程度的研究,但通常以